\chapter{Measures}
\begin{quote}
    Whereof one cannot speak, thereof one must be silent.

    -- Ludwig Wittgenstein
\end{quote}

\section{Introduction}

\section{\textsigma -algebras}
    \begin{note}[page 21]
        If $X$ is any uncountable set, then
        \begin{align*}
            \mathcal{A} = \left\{ E \subset X : \text{$E$ or $E^c$ is countable} \right\}
        \end{align*}
        is a \textsigma-algebra.
        This defenition only makes sense whenever $X$ is uncountable. If $X$ is countable 
        then $\mathcal{A} = \mathscr{P}(X)$ and we know that powerset of any set, 
        is a \textsigma-algebra for that set.
        \begin{proof}
            Let $\{E_j\}_1^\infty \subset \mathcal{A}$ and $E = \bigcup\{E_j\}_1^\infty$; 
            If all $E_j$s are countable, since countable union of coutable sets is countable, 
            $E$ is countable. otherwise there exists at least on $E_k$ that is not countable;
            since $E_k \in \mathcal{A}$, $E_k^c$ is countable.
            \begin{align*}
                E^c = \left(\bigcup_1^\infty E_j\right)^c = \bigcap_1^\infty E_j^c
            \end{align*}
            So $\#E^c \leq \#E_j^c$ for all $j \in \mathbb{N}$.
        \end{proof}
    \end{note}

    \begin{note}[page 22]
        \textsigma-algebra generated by $\varepsilon$:
        \begin{align*}
            \mathscr{M}(\varepsilon) = \bigcap \left\{ \mathscr{M}_\alpha : \text{$\varepsilon \subset \mathscr{M}_\alpha$ and $\mathscr{M}_\alpha$ is a \textsigma-algebra} \right\}
        \end{align*}
    \end{note}

\section{Measures}

\section{Outer Measures}

\section{Borel measures on the Real Line}
